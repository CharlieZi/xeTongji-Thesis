\chapter{引~言}

\section{}

金属编织加强软管在工程中有着非常广泛的应用,其结构如图\ref{intro}所示,是液压传动系统中非常重要的组成部分。因其可以承受相对较大的内压、轴向荷载,同时保持较小的质量、弯曲刚度[1],带来以下优势:可以减小系统的刚度,吸收液压源产生的振动;安装方式灵活,节约了系统内部的空间。
目前对金属编织加强软管的研究,多见于汽车工业中的中刹车管[2]、转向传动管[3]、空调管[4] 等。






\begin{figure}[t]
\begin{tabularx}{5in}{ccc}
\begin{minipage}[t]{2in}
\includegraphics[width=1.5in]{hose-1}
\end{minipage}
\begin{minipage}[t]{2in}
\includegraphics[width=1.5in]{hose-2}
\end{minipage}
\end{tabularx}
\begin{minipage}[t]{2in}
\includegraphics[width=1.5in]{unit-1}
\end{minipage}

\caption{This is a caption}
\label{intro}

\end{figure}







对软管加强层理论的研究,基本使用通过加强层的总体的要是通过软管理论主要有两个分支[5]:一种是加强层含量较低,橡胶管起主要作用的软管,由Kuipers等人[6,7]提出并完善,适用于帘线加强的软管;另一种是编织加强层主要承力的软管,主要研究的是钢丝螺旋缠绕加强层。软管轴向受拉时,缠绕的金属丝会沿缠绕方向“流动”。编织加强层仅作为螺旋缠绕的一种特例:两层缠绕方向相反,且不允许“流动”的缠绕层[5]。
近20年对编织加强结构的研究主要集中在复合材料编织。复合材料纤维编织的与金属纤维编织的传力机制差别非常明[8]:复合材料纤维只承受单向应力;而金属编织层中的金属丝间的接触关系会直接影响编织层整体的传力,不能忽视。因此,并没有至今尚没有成熟、独立,考虑金属丝间接触关系的编织加强层理论。
有学者尝试用连续介质力学的基本理论推导编织层的本构,如Evans[5]编织层金属纤维侧向传力机制,Horgan等人[9] 提出了纤维加强材料的应变能密度函数,国内学者计算了编织结构强度与突加荷载的情况[10,11]。但主流的研究办法还是结合实验,提出能够反映加强层力学行为的有限元模型。Wijaya[4]对包含软管各层材料及编织层的试件进行了压缩实验,认为金属编织层的应力应变关系是线性的,在软管整体动态特性的研究中取得了较好的效果。Cho[3]研究了编织层在扣压安装接头中的力学行为,结合压缩实验提出了弹塑性的本构模型,Rattensperger[8]同样针对压缩的过程,编织层厚度方向引入一组等效非线性弹簧,表征金属纤维间相互作用。
Hachemi[12]对编织层进行了拉伸试验,将复合材料编织中考虑材料非线性行为的特征单元法(首先由Reese[13]提出)引入金属编织层的研究,提出了能够反映编织结构编织角变化的本构模型。该模型将编织层简化两层Rebar单元,只在编织方向上有刚度。但该模型仍然没有考虑两层Rebar单元之间相互接触的关系。
本研究实验表明,Hachemi[12]本构模型的非线性行为并不足够强,不能与本研究中的高压金属编织加强软管的拉伸实验结果相吻合。我们试图通过引入金属纤维间的接触关系来修正该理论与实验的差距。使得包括非线性段的实验结果都能够与修正后的理论值相吻合。



	
	
