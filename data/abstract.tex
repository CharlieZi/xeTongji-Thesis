
\begin{abstract}
基于Hachemi提出的金属编织层本构理论以及拉伸实验的研究方法,进行了不同尺寸金属编织软管的拉伸实验。发现金属编织层在拉伸时会出现强烈的结构非线性,而该理论不能在非线性段与实验结果相吻合。提出了对理论中编织层刚度矩阵修正方法,将金属纤维间交叠产生的接触等效地转化为“修正基体”(Modified Matrix Method),独立于金属纤维自身产生的刚度。同时,结合编织角在实验中呈现的变化趋势,提出了以下假设:管状金属编织层在等位移拉伸荷载下编织角呈线性变化。这是与Hachemi的假设恰好相反的:Hachemi认为编织层受拉时,编织角减小到一定程度时会发生锁定现象,而实验证明了编织角并不会发生锁定,且直到编织层破坏为止,编织角变化的趋势基本呈线性。还提出了修正编织角变化的加速系数k,通过金属纤维间的侧向接触解释了实验与数值仿真结果中,力位移曲线与编织角变化曲线不能同时吻合的力学机理。

we conducted an experiment and put forward several modifications, based on the method proposed in Hachemi’s research of metal wire braid reinforced hose, in order to enhance the constitutive theory, which is discovered not keeping up with the experimental results when it shows far more non-lineal mechanical behavior than the theory predict. We introduced a “modifying matrix”, from composites mechanics, to detach inter-wire contacts from wire elongation, seldom considered before. We also proposed a hypothesis opposite to Hachemi’s: the hose’s braid angle decreased linearly, applied displacement load with constant loading rate, rather than locked at a certain degree. So that we introduce a modification coefficient k, accelerating the decrease of braid angle to match the linearity in force-displacement curve. Lateral contact is considered to be the factor of excessively decreased braid angle when the calculated curve perfectly meet the experimental one, with suitable k.

Key Words:Braid,Hose,modification, 
\end{abstract}


